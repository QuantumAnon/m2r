\documentclass[./m2r-report.tex]{subfile}

%Much of algebra is concerned with substructures of algebraic objects, such as
%subgroups, ideals, subspaces, etc.
%Almost all of these objects have some inherent structure or 'rules' from which
%the existence of these sub-objects can be inferred.
%Ramsey theory aims to find conditions for the existence of substructures when
%the superstructure has very few rules.
%We aim to instil an appreciation of the subject in the reader and show how it may
%be applied to other areas of mathematics.

Algebra is primarily concerned with the structure of algebraic objects and their
substructure (subgroups, ideals, subspaces, etc.).
Algebraic objects satisfy a set of axioms defining their intrinsic
order: existence of ordered substructure can sometimes be inferred through these.
Ramsey theory aims to find conditions for the existence of substructure when the
superstructure satisfies very few rules.

%A typical example of an object with little constraint on its structural
%properties is a graph.
%Graphs are not much more than sets; yet there are many graph theoretical results
%of the following form: a large enough graph exhibits regularity.
%A notable example is Szemerédi’s regularity lemma, which we state as a theorem:  

%\begin{theorem}[Szemerédi]
 % For any $\epsilon >0$, any graph admits an $\epsilon$-regular partition.  
%\end{theorem}

%This theorem is powerful when we consider large, dense graphs (\cite{Diestel2000});
%$\epsilon$-regularity means in this case that the distribution of edges in our
%graph is in some sense homogeneous, and thus “almost random”. This result shows
%how a class of seemingly “unstructured” objects can exhibit some strong
%structural properties.   

The original Ramsey theorem is a result about classes and their partitions. 
We adapt the statement from Ramsey's original paper:

\begin{theorem}[\textcite{Ramsey1930}]
  Let $\Gamma$ be a countably infinite set. For all $r,k\in\mathbb{Z}^+$, for 
  any $r$-partitioning, $\{\mathcal{C}_{1},\cdots,\mathcal{C}_{r}\}$, of the set
  of $k$-combinations of $\Gamma$ there exists an infinite subset
  $\Delta\subset\Gamma$ such that:
  \begin{center}
  $\exists i\in \{1,\cdots,r\}$ such that every $k$-combination of $\Delta$ is
  in $\mathcal{C}_{i}$.
  \end{center}
\end{theorem}

Now, sets are a typical example of mathematical objects with little constraint 
on their structural properties. In this sense, Ramsey's theorem shows
how a class of seemingly “unstructured” objects can exhibit some strong
structural properties.

Ramsey theory can be seen as a generalisation of this approach to the study of
unstructured or loosely structured mathematical objects.
In Ramsey theoretical terms, a structure exhibits regularity if, no matter how
it is partitioned, there exists a substructure entirely contained in one element
of the partition.
Partitioning is a natural operation that can be applied to a vast class of
objects; hence the scope of the theory is broad.

 

We aim to instill an appreciation of the subject in the reader and to show how
it may be applied to a wide range of mathematical structures. We will also focus
on the ability of the theory to act as a bridge between combinatorics, algebra,
geometry, and topology.

% The condition is often a question of quantity, i.e. a substructure will exist
% in every large enough object.

 %Local Variables:
 %TeX-master: "m2r-report.tex"
 %End:

% LocalWords:  Ramsey
