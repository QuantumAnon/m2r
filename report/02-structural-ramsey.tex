\documentclass[../main.tex]{subfile}

The original Ramsey theorem states the existence of a complete monochromatic
subgraph in any two-colouring of a sufficiently large graph.
Can this statement be extended to other classes of graphs?
A useful structure for exploring this question is the random graph, which is an
example of a Fraïssé limit, a construction that can be applied to many
different structures.
In this section we look at how the random graph is constructed, some of its
properties, and how its existence can be used to infer some Ramsey type results.

We begin with a look at relational languages and how they apply to finite graphs.

\subsection{Relational Languages}
\label{sec:2.1}

\subsection{The Random Graph}
\label{sec:2.2}

\subsection{The Ramsey Property for Structures}
\label{sec:2.3}

% LocalWords:  Ramsey subgraphs subgraph Fraïssé
