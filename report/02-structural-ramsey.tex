\documentclass[../m2r-report.tex]{subfile}

The original Ramsey theorem states the existence of a complete monochromatic
subgraph in any two-colouring of a sufficiently large graph.
Can this statement be extended to other classes of graphs or objects?
A useful structure for exploring this question is the random graph.
This is an example of a Fraïssé limit, a construction that can be applied to many
different structures.
In this section we look at how the random graph is constructed, some of its
properties, and how its existence can be used to infer some Ramsey type results.

We begin with the definition of some model theoretic concepts that are needed to
state the generalized form of Ramsey Theorem.

\subsection{Structures, Ages and Amalgamation Classes}
\label{sec:2.1}

Let $I$ be an index set, we call $\Delta = (\delta_i \in \Z^+)_{i\in I}$ a
type and the pair $L=(I,\Delta)$ a language.
Given a set $A$, an $n$-ary relation is a subset of $A^n$.
An $L$-structure $\str{A}$ is a pair $(A,(\rel{i}{\str{A}})_{i\in I})$ where each
$\rel{i}{\str{A}}$ is a $\delta_i$-ary relation, i.e. a subset of $A^{\delta_i}$
\cite{Hubicka2015}.
If $A$ is finite, the structure $\str{A}$ is said to be finite, with a similar
statement holding for countable sets.
An example is the language of finite graphs $L = (\{1\}, \delta_1 = 2)$ where
each structure $\str{A}$ is a graph over the set $A$.

Homomorphisms between structures over the same language are maps that preserve
relations, so $f : \str{A} \to \str{B}, \rel{i}{\str{A}} \mapsto
\rel{i}{\str{B}}, \forall i \in I$.
An injective homomorphism where $f^{-1} : \rel{i}{\str{B}} \mapsto
\rel{i}{\str{A}}, \forall i \in I$ is called an embedding, an isomorphism of
structures is a surjective embedding\cite{Hubicka2015}.

We say that $\str{B}$ is a substructure of $\str{A}$ if the inclusion $B
\subseteq A$ is an embedding of structures.
We denote the set of substructures of $\str{A}$ isomorphic to $\str{B}$ by
$\sstr{A}{B}$\cite{Hubicka2016}.

\begin{definition}[Amalgamation Class\supercite{Hubicka2016}]
  A class $\class{K}$ is an amalgamation class if it satisfies the following
  properties:
  
  \begin{itemize}
  \item \textbf{Hereditary Property}: For every $\str{A} \in \class{K}$ and every
    substructure $\str{B} \subseteq \str{A}, \str{B}\in\class{K}$ 
  \item \textbf{Joint Embedding Property}: For every $\str{A}, \str{B} \in
    \class{K}$, there exists $\str{C} \in \class{K}$ such that both $\str{A}$ and
    $\str{B}$ embed in $\str{C}$
  \item \textbf{Amalgamation Property}: For every $\str{A}, \str{B}_1, \str{B}_2
    \in \class{K}$ with $\str{A}$ embedding in both $\str{B}_1$ and $\str{B}_2$,
    there exists $\str{C} \in
    \class{K}$ such that $\str{B}_1$ and $\str{B}_2$
    both embed into $\str{C}$ and the diagram below commutes\cite{Cameron1990}.
  \end{itemize}
  \begin{center}
    \adjustbox{scale=1.5}{%
      \begin{tikzcd}
        A
        \arrow[r, hookrightarrow, "\alpha_1"]
        \arrow[d, hookrightarrow, "\alpha_2"]&
        B_1\arrow[d, dashed, hookrightarrow, "\exists\beta_1"]\\
        B_2\arrow[r, dashed, hookrightarrow, "\exists\beta_2"]& C
      \end{tikzcd}
    }
  \end{center}
\end{definition}

\begin{definition}[Age\supercite{Cameron1990,Macpherson2011}]
  The age of a countable structure, say $\str{A}$, is the class of finite
  $L$-structures, $\str{B}$, such that $\str{B}$ is isomorphic to
  a substructure of $\str{A}$.
  It is reasonably clear that $\age\str{A}$ has the hereditary property, 
  $\age\str{A}$ also has the joint embedding property as $\str{B}_1, \str{B}_2
  \in \age\str{A} \implies \str{B}_1, \str{B}_2$ embed in $\str{A} \in \age\str{A}$.
\end{definition}

Two natural questions to ask now are whether the age of a particular structure
is an amalgamation class, and whether a given amalgamation
class is the age of some structure.
The correspondence between the two is given by Fraïssé's theorem which makes use
of the concept of homogeneity.

A structure $\str{A}$ is homogeneous if for every pair of isomorphic
substructures $\str{U}, \str{V} \subseteq \str{A}, f : \str{U} \to \str{V}$ the
isomorphism, there exists an automorphism $\tilde{f} \in\aut\str{A}$ which
extends $f$, i.e. $\tilde{f} : \str{U} \mapsto \str{V}$\cite{Macpherson2011}.

We now state the main theorem of this subsection

\begin{theorem}[Fraïssé\supercite{Fraisse1953}]
  Let $L$ be a language
  \begin{itemize}
  \item Given a homogeneous $L$-structure $\str{M}$, $\age\str{M}$ is an
    amalgamation class.
  \item Given a non-empty amalgamation class of finite structures $\class{C}$,
    there exists a homogeneous $L$-structure $\str{M}$ such that $\age\str{M}
    = \class{C}$.
    This structure is unique up to isomorphism
  \end{itemize}
\end{theorem}

We omit a proof of this theorem, but a sketch is given in \textcite[1602]{Macpherson2011}.
We call the structure given by the second point of the theorem the Fraïssé limit
of the class $\class{C}$.

\subsection{Ramsey Classes}
\label{sec:2.4}
The original statement of Ramsey's theorem is, broadly speaking, a statement
about the existence of a sub-object in any partition of a 'large-enough' object.
We now introduce the idea of a Ramsey class which applies this to classes.

\begin{definition}[Ramsey Class\supercite{Nesetril1995}]
  A structure $\str{C}$ is said to be $(t,\str{A})$ Ramsey for a structure
  $\str{B}$ if every $t$-colouring of the set $\sstr{C}{A}$ contains
  $\str{B'}: \str{C}\supseteq\str{B'} \simeq \str{B}$ such that $\sstr{B'}{A}$ is
  monochromatic.
  To make this more concise, we write $\str{C}\ramsArrow{\str{B}}{\str{A}}{t}$.

  A class $\class{K}$ is said to have the $\str{A}$-Ramsey property if for every
  $\str{B}\in\class{K}$ and $t\in\Z^+$, there exists $\str{C}\in\class{K}$ such that
  $\str{C}\ramsArrow{\str{B}}{\str{A}}{t}$.

  If $\class{K}$ has the $\str{A}$-Ramsey property for every $\str{A}\in
  \class{K}$, then $\class{K}$ is a Ramsey class.
\end{definition}

A concrete example can be given in terms of finite complete graphs.
The well-known fact that any 2-colouring of $K_6$ contains a monochromatic
$K_3$ translates to the statement $K_6\ramsArrow{K_3}{K_2}{2}$.
Due to the generalised Ramsey theorem we can even state that the class of finite
complete graphs has $K_2$ Ramsey property.
Since for any $K_n$ and any $t$ a positive integer, there exists a $K_m$ such
that any $t$-colouring of the $K_2$ substructures of $K_m$, the edges, 

\subsection{The Random Graph}
\label{sec:2.3}
% LocalWords:  Ramsey subgraphs subgraph Fraïssé surjective

% Local Variables:
% TeX-master: "m2r-report.tex"
% End:
