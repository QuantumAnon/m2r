\documentclass[../m2r-report.tex]{subfile}

As we've seen, Ramsey theory has manifestations in areas as diverse as graph
theory, combinatorics, and geometry.
The field of Structural Ramsey theory aims to use model theory to study this
generality and the implications of applying Ramsey type results to different
objects.
An object that demonstrates many of the results in Structural Ramsey theory is
the Random (or Rado) graph.
It is an example of a Fraïssé limit, which is a construction found in
model theory.
Results from Structural Ramsey theory can then be used to state properties of
the Random graph.

This section will focus on the preliminaries behind the construction of Fraïssé
limits, the existence and uniqueness of the Random graph, and some of the
Ramsey type properties that it exhibits.

\subsection{Structures, Ages and Amalgamation Classes}

We begin with a few preliminaries and some important results about them.

\begin{definition}[Structures over a Relational Language\cite{Hubicka2015}]
  Let $I$ be a finite index set.
  We call the pair $L=(I,\Delta)$ a \textit{language}, where $\Delta$ is an
  $I$-tuple of positive integers, called the \textit{type}.
  
  Given a set $A$, an $n$-ary relation is a subset of $A^n$.
  An \textit{$L$-structure} $\str{A}$ is a pair $(A,\rel{A})$, where
  $\rel{A}=(\rel[i]{\str{A}})_{i\in I}$ is an $I$-tuple of $\delta_i$-ary
  relations, i.e. $\forall i \in I, \rel[i]{A}\subseteq A^{\delta_i}$.

  If $A$ is finite, the structure $\str{A}$ is said to be finite, with a similar
  statement holding for countable sets.
  In general, the cardinality of a structure is defined to be the cardinality of
  the set on which it is defined.
\end{definition}

An example of a language is $L=(\{1\}, (2))$.
An example of a structure on that language is $\str{A} = (A, \rel[<]{A})$, where
$A$ is any non-empty finite subset $A \subset \Z^+$, equipped with the relation
$\rel[<]{A} = \{(x,y): x < y\} \subset A^2$. 
This forms a class of structures that we will call $\flo$, the class of
finite linear orderings.
Any $\str{A}\in\flo$ must satisfy the following properties
\begin{itemize}
\item\textbf{Transitive}: $\forall x, y, z \in A : (x,y)\in\rel[<]{A}$ and
  $(y,z)\in\rel[<]{A} \implies (x,z)\in\rel[<]{A}$
\item\textbf{Semiconnex}: $\forall x, y \in A : x \neq y \implies (x,y) \in
  \rel[<]{A}$ or $ (y,x) \in \rel[<]{A}$
\item\textbf{Irreflexive}: $\forall x, y \in A : (x,y)\in\rel[<]{A} \implies
  (y,x)\notin\rel[<]{A}$
\end{itemize}

\begin{definition}[Homomorphisms and Substructures\cite{Hubicka2016,Hubicka2015}]
  \textit{Homomorphisms} between structures over the same language are maps that
  preserve relations,\\
  so $f : \str{A} \to \str{B}$ such that $\forall i \in I, f(\rel[i]{A})\subseteq
  \rel[i]{B}$.

  An injective homomorphism where $f^{-1}(\rel[i]{\str{B}}) \subseteq
  \rel[i]{\str{A}}, \forall i \in I$ is called an \textit{embedding}, an
  \textit{isomorphism} of structures is a surjective embedding.

  If for two structures $\str{A},\str{B}$, the inclusion $B \subseteq A$ is an
  embedding, we call $\str{B}$ an \textit{induced substructure} of $\str{A}$,
  though we will shorten this to \textit{substructure}.
  We denote the set of substructures of $\str{A}$ isomorphic to $\str{B}$ by
  $\sstr{A}{B}$.
\end{definition}

Returning to $\flo$, there are many bijections between subsets of $\Z^+$ of
the same cardinality, but this may not always preserve the relation.
To show that there is an isomorphism between any two $\str{A}, \str{B} \in
\flo$ with $\card{\str{A}} = \card{\str{B}}$, one can proceed by induction (show
that there is an isomorphism between $\str{A} : \card{\str{A}} = n$ and
$\str{N}_n = (\{1,\dots,n\},\{(1,2),\dots,(1,n),(2,3),\dots,(n-1,n)\})$).
With this we can say that there are only countably many structures in
$\flo$ up to isomorphism.

$\flo$ is also an example of an amalgamation class, members of which
exhibit a few more properties.

\begin{definition}[Amalgamation Class\cite{Cameron1990,Hubicka2016}]
  A class $\class{K}$ is an \textit{amalgamation class} if it satisfies the
  following properties:
  
  \begin{itemize}
  \item \textbf{Hereditary Property}: For every $\str{A} \in \class{K}$ and every
    substructure $\str{B} \subseteq \str{A}, \str{B}\in\class{K}$ 
  \item \textbf{Joint Embedding Property}: For every $\str{A}, \str{B} \in
    \class{K}$, there exists $\str{C} \in \class{K}$ such that both $\str{A}$ and
    $\str{B}$ embed in $\str{C}$
  \item \textbf{Amalgamation Property}: For every $\str{A}, \str{B}_1, \str{B}_2
    \in \class{K}$ with $\str{A}$ embedding in both $\str{B}_1$ and $\str{B}_2$,
    there exists $\str{C} \in
    \class{K}$ such that $\str{B}_1$ and $\str{B}_2$
    both embed into $\str{C}$ and the diagram below commutes.
  \end{itemize}
  \begin{center}
    \adjustbox{scale=1.5}{%
      \begin{tikzcd}
        A
        \arrow[r, hookrightarrow, "\alpha_1"]
        \arrow[d, hookrightarrow, "\alpha_2"]&
        B_1\arrow[d, dashed, hookrightarrow, "\exists\beta_1"]\\
        B_2\arrow[r, dashed, hookrightarrow, "\exists\beta_2"]& C
      \end{tikzcd}
    }
  \end{center}
\end{definition}

\begin{example}
  $\flo$ is an amalgamation class

  \begin{proof}
    Prove properties, suppose $\str{B_1},\str{B_2}\in\flo$.
    \begin{itemize}
    \item\textbf{HP}: Given $B_1 \subseteq B_2$, we
      have $\card{\str{B_1}} \leq \card{\str{B_2}}$. Every $\str{A}\in\flo :
      \card{\str{A}} = m$ is isomorphic to $\str{N}_m$ as defined above and
      there is a clear embedding of $\str{N}_m$ into $\str{N}_n$ where $m \leq
      n$.
    \item\textbf{JEP}: Suppose again that $\card{\str{B_1}} = m,
      \card{\str{B_2}} = n$.
      $\str{B_1}$ embeds into $\str{N}_{m+n}$ by mapping to the values
      $1,\dots,m$ and $\str{B_2}$ also embeds by mapping to the values $m+1,\dots,n$.
    \item\textbf{AP}: Suppose that there is $\str{A}$ which embeds into both
      $\str{B_1}$ and $\str{B_2}$ by $\alpha_1, \alpha_2$ respectively.
      Partition $B_1, B_2$ into three classes each:
      \begin{itemize}
      \item The $x\in B_i$ such that $\exists y \in A$ with
        $(x,\alpha_i(y)) \in \rel[<]{B_i}$, call these sets $B^-_i$.
      \item The $x\in B_i$ such that $\exists y \in A$ with
        $x=\alpha_i(y)$, call these sets $B^A_i$.
      \item The $x\in B_i$ such that $\exists y \in A$ with
        $(\alpha_i(y),x) \in \rel[<]{B_i}$, call these sets $B^+_i$.
      \end{itemize}
      Define, $n_1 = \card{B^-_1}, n_2 = n_1 + \card{B^-_2}, n_3 = n_2 + \card{A},
      n_4 = n_3 + \card{B^+_2}, n_5 = n_4 + \card{B^+_2}$ and let $C = \str{N}_{n_5}$.
      We define the embeddings as follows: 
      \begin{align*}
        \beta_1:&\str{B_1}\to\str{C},\\
                &B^-_1\mapsto\{1,\dots,n_1\},\\
                &B^A_1\mapsto\{n_2+1,\dots,n_3\},\\
                &B^+_1\mapsto\{n_3+1,\dots,n_4\}\\
        \beta_2:&\str{B_2}\to\str{C},\\
                &B^-_2\mapsto\{n_1,\dots,n_2\},\\
                &B^A_2\mapsto\{n_2+1,\dots,n_3\},\\
                &B^+_2\mapsto\{n_4+1,\dots,n_5\}
      \end{align*}
      It is then clear that $\beta_1\circ\alpha_1 = \beta_2\circ\alpha_2$.
    \end{itemize}
  \end{proof}
\end{example}

Another useful construction is the age of a structure.

\begin{definition}[Age\cite{Cameron1990,Macpherson2011}]
  The \textit{age} of a countable structure, say $\str{A}$, is the class of
  finite $L$-structures, $\str{B}$, such that $\str{B}$ is isomorphic to
  a substructure of $\str{A}$.
\end{definition}

It is reasonably clear that when $\str{A}$ is a countable relational structure,
$\age\str{A}$ has the hereditary property.
By a proof similar to the one given for finite linear orders, it can also be
shown that $\age\str{A}$ has the JEP.

Two natural questions to ask now are whether the age of a particular structure
is an amalgamation class, and whether a given amalgamation
class is the age of some structure.
The correspondence between the two is given by Fraïssé's theorem which makes use
of the concept of homogeneity.

\begin{definition}[Homogeneous Structures\cite{Macpherson2011}]
  A structure $\str{A}$ is \textit{homogeneous} if for every pair of finite
  isomorphic substructures $\str{U}, \str{V} \subseteq \str{A}$, with $f :
  \str{U} \to \str{V}$ being the isomorphism, there exists an automorphism
  $\tilde{f} \in\aut\str{A}$ which extends $f$, i.e. $\tilde{f}$ restricted to
  $\str{U}$ is $f$.
\end{definition}

An example of homogeneity is given by the rationals with the order relation.
The argument below is taken from \textcite[1601]{Macpherson2011}.
\begin{example}
  $\str{Q} = (\mathbb{Q},<)$ is homogeneous.
  \begin{proof}
    Let $\str{U}, \str{V} \subseteq \str{Q}$ be an isomorphic pair of
    finite substructures with isomorphism $f$.
    Since they are both finite, they both have a minimum and maximum.
    Let $a \in \mathbb{Q}\setminus U$, since $\str{U}$ is finite either:
    \begin{enumerate}
    \item $\forall u \in U, a < u$, this is equivalent to $a < u\sumin$
    \item $\exists u_1, u_2 \in U, u_1 < a < u_2$
    \item $\forall u \in U, u < a$, this is equivalent to $u\sumax < a$
    \end{enumerate}
    Construct $\tilde{f}$ such that $\tilde{f} : (-\infty,u\sumin) \mapsto
    (-\infty,f(u\sumin)), (u_i,u_{i+1}) \mapsto (f(u_i),f(u_{i+1})),
    (u\sumax,\infty) \mapsto (f(u\sumax),\infty)$ where the bounded intervals
    are mapped via a linear interpolation and the unbounded intervals are mapped
    linearly.
    This is an automorphism on $\str{Q}$.
  \end{proof}
\end{example}

We now state the main theorem of this subsection

\begin{theorem}[Fraïssé\cite{Fraisse1953}\label{thr:fraisse}]
  Let $L$ be a language,
  \begin{itemize}
  \item Given a homogeneous $L$-structure $\str{M}$, $\age\str{M}$ is an
    amalgamation class.
  \item Given a countable non-empty amalgamation class of finite structures
    $\class{C}$, there exists a unique countable homogeneous $L$-structure
    $\str{M}$ such that $\age\str{M} = \class{C}$.
  \end{itemize}
\end{theorem}

We omit a proof of this theorem, but a sketch is given in \textcite[1602]{Macpherson2011}.
We call the structure given by the second point of the theorem the Fraïssé limit
of the class $\class{C}$.

A good example of a Fraïssé limit is that of $\flo$, which has limit $\str{Q}$.
It is clear that any finite linear order can be embedded into $\str{Q}$, so it
only remains to show that any finite substructure of $\str{Q}$ is isomorphic to
some structure in $\flo$.
This can be done by induction in a similar manner to showing that there are only
countably many structures in $\flo$ up to isomorphism.

\subsection{The Random Graph}

The defining property of the random graph is

\begin{remark}[Extensibility Property (EP)\cite{Cameron2013}\label{rem:extprop}]
  Given a finite set of vertices in a graph $U = \{u_1, \dots, u_m\} \subseteq
  G_V$, we define $\adj{U}$ as the set of vertices in $G_V$ adjacent to every
  element in $U$. 

  
  A graph $G = (G_V,G_E)$ has the \textit{extensibility property} if given two
  finite sets of distinct vertices $U = \{u_1,\dots,u_m\}, V =
  \{v_1,\dots,v_n\} \subseteq G_V$, there exists a vertex $z\in
  G_V\setminus(U\cup V)$ such that $z \in \adj{U}$ and $z \notin \adj{V}$.
\end{remark}

\textcite{Erdos1963} show that by taking a countable set of vertices and
assigning edges between vertices with probability $\frac{1}{2}$, one produces a
graph that almost surely has the extensibility property.
While the original paper gives a proof, we reproduce a proof by
\textcite[1-2]{Cameron2013}.

\begin{prop}
  We define a countable random graph as a countable structure $\str{G}=(G,R)$
  which satisfies
  \begin{itemize}
  \item $\forall x\in G, \neg((x,x)\in R)$
  \item $\forall x\in G,\forall y\in G,(x,y)\in R \implies (y,x)\in R$
  \item $\forall x\in G,\forall y\in G, \prob((x,y)\in R)=\frac{1}{2}$
  \end{itemize}

  Then any countable random graph almost surely has the \hyperlink{rem:extprop}{EP}.

  \begin{proof}
    We show that the set of countable graphs that do not have the
    \hyperlink{rem:extprop}{EP} has probability 0, i.e. it is a null set.
    We use an elementary lemma from measure theory that states a countable union
    of null sets is also null, which follows from countable additivity of a
    measure.
    Since we only have countably many choices of $m$ and $n$, there are only
    countably many choices of $U = \{u_1,\dots,u_m\}$ and $V =
    \{v_1,\dots,v_n\}$ so we only need to consider a particular choice of $U$
    and $V$ and show that the set of graphs that don't have the
    \hyperlink{rem:extprop}{EP} for that choice is null in the limit.

    Suppose that $Z = \{z_1,\dots,z_N\}$ are distinct vertices such that each
    $z_i$ does not satisfy the \hyperlink{rem:extprop}{EP}.
    For any particular $z_i$, the probability of this occurring is $1 - Pr(z_i \in
    \adj U \text{ and } z_i \notin \adj V) = 1 - \frac{1}{2^{m+n}}$
    (adjacency is a union of independent events).
    The probability of this occurring for every $z_i \in Z$ is $\left(1 -
      \frac{1}{2^{m+n}}\right)^N$.
    This approaches 0 as $N\to\infty$ so we can state that every countable
    random graph fails \hyperlink{rem:extprop}{EP} with probability 0 $\implies$ it
    satisfies it with probability 1.
  \end{proof}
\end{prop}

Note that the fixed probability of an edge occurring could be replaced with any
$p\in(0,1)$ and the proof would be identical.

\begin{prop}\label{prop:EPIso}
  If two countable graphs both satisfy the \hyperlink{rem:extprop}{EP}, then they
  are isomorphic.

  \begin{proof}
    Let $\str{\Gamma_1} = (\Gamma_1,\rel{\Gamma_1})$ and $\str{\Gamma_2} =
    (\Gamma_2,\rel{\Gamma_2})$ be two graphs that satisfy the
    \hyperlink{rem:extprop}{EP}.

    Suppose we have an isomorphism $f$ between two finite substructures
    $\str{G_1}\subset\str{\Gamma_1},\str{G_2}\subset\str{\Gamma_2}$, these are
    induced subgraphs which necessarily have the same cardinality $n$, and we
    take an additional vertex $x_{n+1} \in \Gamma_1\setminus G_1$.
    We can extend this to an isomorphism between the induced subgraphs on $G_1'
    = G_1 \cup \{x_{n+1}\}$ and $G_2' = G_2 \cup \{y_{n+1}\}$ using the
    \hyperlink{rem:extprop}{EP} in the following manner.

    We can partition $G_1$ into $U = \{x \in G_1 : x \in \adj\{x_{n+1}\}\},
    V = \{x \in G_1 : x \notin \adj\{x_{n+1}\}\}$ and take the images of these
    classes under $f$, say $U', V'$.
    By the \hyperlink{rem:extprop}{EP}, we have a $y_{n+1} \in \Gamma_2\setminus G_2
    : y_{n+1} \in \adj{U'}$ and $y_{n+1} \notin \adj{V'}$, so we can extend
    the isomorphism by setting $f'(x_{n+1}) = y_{n+1}$.
    It is clear to see that this is an isomorphism between substructures.
    Furthermore, note that we could switch $\str{\Gamma_1}$ and $\str{\Gamma_2}$
   and the proof would be identical.

   We can now begin the construction of the full isomorphism.
   Since $\Gamma_1, \Gamma_2$ are countable they can be enumerated as $\Gamma_1
   = \{x_1,x_2,\dots\}, \Gamma_2=\{y_1,y_2,\dots\}$.
   We construct our isomorphism as follows, start with $f_0$ being the empty
   map and proceed by induction.
   Suppose that the $n$th isomorphism has been constructed.
   If $n$ is even, pick the $x_i\in\Gamma_1$ with the smallest index not in the
   domain of $f_n$, and find the $y_j\in\Gamma_2$ with the least index which
   extends the isomorphism in the manner above, call the result $f_{n+1}$.
   If $n$ is odd, repeat the process but instead find the $y_i\in\Gamma_2$ with
   the least index not in the range of $f_n$, and find the $x_j\in\Gamma_1$ with
   the smallest index which will extend $f_n$.
   
   Take the countable union of these partial maps and call the result $f$,
   We have a countable enumeration of $\Gamma_1$ given by how elements were
   added to the domain of $f$ so we can be sure that every element in $\Gamma_1$
   is mapped to a unique element of $\Gamma_2$.
   We can also enumerate $\Gamma_2$ by how elements of the range were added, so
   every element in $\Gamma_2$ has unique pre-image.
   Thus $f$ is a bijection, and by our construction, an isomorphism.
 \end{proof} 
\end{prop}

With the above two propositions, we have shown that there exists a countable
structure that satisfies the \hyperlink{rem:extprop}{EP} and that it is unique up to
isomorphism.
We shall call this structure the random graph, $\rand$.

From the proof of \autoref{prop:EPIso}, it can be deduced that if we have two
finite isomorphic substructures of $\rand$, we can extend the partial
isomorphism to an automorphism of the entire graph, i.e. $\rand$ is homogeneous.
By~\autoref{thr:fraisse}, $\age{\rand}$ must be an amalgamation class over the
same language as $\rand$.
The final property we will show about the random graph is that it is the Fraïssé
limit of the class of undirected finite graphs without loops, $\gra$.

We define $\gra$ as follows.
$\str{A}\in\gra$ is a finite structure over the language $(1,(2))$ satisfying the
following properties.
\begin{itemize}
\item\textbf{Irreflexive}: $\forall x \in A, \neg((x,x)\in\rel{A})$
\item\textbf{Symmetric}: $\forall x, y\in A, (x,y)\in\rel{A} \implies (y,x)\in\rel{A}$
\end{itemize}

\begin{prop}
  The Fraïssé limit of $\gra$ is $\rand$.

  \begin{proof}
    We proceed by double inclusion.

    It is clear that every finite substructure of $\rand$ is a member of $\gra$,
    i.e. $\age\rand\subseteq\gra$, since the two defining properties of $\gra$
    are satisfied by $\rand$ by definition.

    So it remains to show that every $\str{A}\in\gra$ embeds into $\rand$, which
    implies that $\gra\subseteq\age\rand$.
    This is achieved in a similar fashion to the proof of~\autoref{prop:EPIso}.
    Enumerate the vertices of $\str{A}$ as $A=\{x_1,\dots,x_m\}$ and let $f_0$
    be the empty map, $D_0 = \emptyset$ its domain.

    Proceed by induction, let $x_{n+1}\in A\setminus D_n$ be the element with
    the least index.
    We can partition the domain of $f_n$ into two finite sets, $U = \{x \in D_n :
    x \in \adj\{x_{n+1}\}\}$ and $V = \{x \in D_n : x \notin
    \adj\{x_{n+1}\}\}$.
    Note that by symmetry we have $x_{n+1}\in\adj U$ and $x_{n+1}\notin\adj V$.
    
    We then extend the homomorphism via the~\hyperlink{rem:extprop}{EP} in the
    usual manner to get $f_{n+1}:D_n\cup\{x_{n+1}\}\to\rand$ a homomorphism.
    Note further than since $\str{A}$ is finite, this process will terminate.
    After which we will have found a substructure of $\rand$ which is isomorphic
    to $\str{A}$ implying that $\gra\subseteq\age\rand$.

    Thus $\gra=\age\rand$

  \end{proof}
\end{prop}


\subsection{Ramsey Classes}
\label{sec:4.4}
The original statement of Ramsey's theorem is, broadly speaking, a statement
about the existence of a sub-object in any partition of a 'large-enough' object.
We now introduce the idea of a Ramsey class which applies this to classes of
structures.

\begin{definition}[Ramsey Class\cite{Nesetril1995}]
  A structure $\str{C}$ is said to be $(t,\str{A})$ Ramsey for a structure
  $\str{B}$ if every $t$-colouring of the set $\sstr{C}{A}$ contains
  $\str{B'}: \str{C}\supseteq\str{B'} \simeq \str{B}$ such that $\sstr{B'}{A}$ is
  monochromatic.
  To make this more concise, we write $\str{C}\radoArrow{\str{B}}{\str{A}}{t}$.

  A class $\class{K}$ is said to have the $\str{A}$-Ramsey property if for every
  $\str{B}\in\class{K}$ and $t\in\Z^+$, there exists $\str{C}\in\class{K}$ such that
  $\str{C}\radoArrow{\str{B}}{\str{A}}{t}$.

  If $\class{K}$ has the $\str{A}$-Ramsey property for every $\str{A}\in
  \class{K}$, then $\class{K}$ is a Ramsey class.
\end{definition}

A concrete example can be given in terms of finite complete graphs.
The well-known fact that any 2-colouring of $K_6$ contains a monochromatic
$K_3$ translates to the statement $K_6\radoArrow{K_3}{K_2}{2}$.
Due to the generalised Ramsey theorem (\ref{thr:genRamsey}), we can state that
the class of finite complete graphs has $K_2$-Ramsey property.
In fact, a direct corollary of the finite Ramsey theorem from the original paper
\cite[267]{Ramsey1930} is that the class of finite complete graphs is a Ramsey
class.

\textit{yet to write}
\begin{itemize}
\item Mention Ramsey + JEP => AP
\item Automorphisms of the random graph (Cameron2013; Cameron1990)
\item Define Strong Amalgamation Property and Extreme Amenability 
\item Theorems 4.7/8 from Kechris2005 
\item Conclusion, apply Ramsey Class property to random graph (nice result
  about groups and colouring)
\end{itemize}

% LocalWords:  Ramsey subgraphs subgraph Fraïssé surjective

% Local Variables:
% TeX-master: "m2r-report.tex"
% End:
