\documentclass[../m2r-report.tex]{subfile}

The original Ramsey theorem states the existence of a complete monochromatic
subgraph in any two-colouring of a sufficiently large graph.
Can this statement be extended to other classes of graphs?
A useful structure for exploring this question is the random graph.
This is an example of a Fraïssé limit, a construction that can be applied to many
different structures.
In this section we look at how the random graph is constructed, some of its
properties, and how its existence can be used to infer some Ramsey type results.

We begin with the definition of some model theoretic concepts that are needed to
state the generalized form of Ramsey Theorem.

\subsection{Structures, Ages and Amalgamation Classes}
\label{sec:2.1}

Let $I$ be an index set, we call $\Delta = (\delta_i \in \Z^+)_{i\in I}$ a
type and the pair $L=(I,\Delta)$ a language.
Given a set $A$, an $n$-ary relation is a subset of $A^n$.
An $L$-structure $\str{A}$ is a pair $(A,(\rel{i}{\str{A}})_{i\in I})$ where each
$\rel{i}{\str{A}}$ is a $\delta_i$-ary relation, i.e. a subset of $A^{\delta_i}$
\cite[739]{Hubicka2015}.
If $A$ is finite, the structure $\str{A}$ is said to be finite, with a similar
statement holding for countable sets.

Homomorphisms between structures over the same language are maps that preserve
relations, so $f : \str{A} \to \str{B}, \rel{i}{\str{A}} \mapsto
\rel{i}{\str{B}}, \forall i \in I$.
An injective homomorphism where $f^{-1} : \rel{i}{\str{B}} \mapsto
\rel{i}{\str{A}}, \forall i \in I$ is called an embedding, an isomorphism of
structures is a surjective embedding\cite[739]{Hubicka2015}.

We say that $\str{B}$ is a substructure of $\str{A}$ if the inclusion $B
\subseteq A$ is an embedding of structures.
We denote the set of substructures of $\str{A}$ isomorphic to $\str{B}$ by
$\sstr{A}{B}$\cite[6]{Hubicka2016}.

A particular type of class that is of interest is the amalgamation class.
This is a class, $\class{A}$, of finite structures that fulfils the following
properties\cite[6]{Hubicka2016}:
\begin{itemize}
\item \textbf{Hereditary Property}: For every $\str{A} \in \class{A}$ and every
  substructure $\str{B} \subseteq \str{A}, \str{B}\in\class{A}$ 
\item \textbf{Joint Embedding Property}: For every $\str{A}, \str{B} \in
  \class{A}$, there exists $\str{C} \in \class{A}$ such that both $\str{A}$ and
  $\str{B}$ embed in $\str{C}$
\item \textbf{Amalgamation Property}: For every $\str{A}, \str{B}_1, \str{B}_2
  \in \class{A}$ with $\str{A}$ embedding in both $\str{B}_1$ and $\str{B}_2$,
  there exists $\str{C} \in
  \class{A}$ such that $\str{B}_1$ and $\str{B}_2$
  both embed into $\str{C}$ and the diagram below commutes\cite[32]{Cameron1990}.
\end{itemize}
\begin{center}
  \adjustbox{scale=1.5}{%
    \begin{tikzcd}
      A
      \arrow[r, hookrightarrow, "\alpha_1"]
      \arrow[d, hookrightarrow, "\alpha_2"]&
      B_1\arrow[d, dashed, hookrightarrow, "\exists\beta_1"]\\
      B_2\arrow[r, dashed, hookrightarrow, "\exists\beta_2"]& C
    \end{tikzcd}
  }
\end{center}

The age of a countable structure, say $\str{A}$, is the class of finite
$L$-structures, $\str{B}$, such that $\str{B}$ is isomorphic to
a substructure of $\str{A}$\cite[32]{Cameron1990}\cite[1602]{Macpherson2011}.
It is reasonably clear that $\age(\str{A})$ has the hereditary property, however,
$\age(\str{A})$ also has the joint embedding property as $\str{B}_1, \str{B}_2
\in \age{\str{A}} \implies \str{B}_1, \str{B}_2$ embed in $\str{A} \in \age{\str{A}}$.

Two natural questions to ask now are whether the age of a particular structure
is an amalgamation class, and whether a given amalgamation
class is the age of some structure.
The correspondence between the two is given by Fraïssé's theorem which makes use
of the concept of homogeneity.

\subsection{Ramsey Classes}
\label{sec:2.4}

\subsection{The Random Graph}
\label{sec:2.3}
% LocalWords:  Ramsey subgraphs subgraph Fraïssé surjective

% Local Variables:
% TeX-master: "m2r-report.tex"
% End:
