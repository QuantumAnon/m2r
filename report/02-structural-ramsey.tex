\documentclass[../m2r-report.tex]{subfile}

As we've seen, Ramsey theory has manifestations in areas as diverse as graph
theory, combinatorics, and geometry.
The field of Structural Ramsey theory aims to use model theory to study this
generality and the implications of applying Ramsey type results to different
objects.
An object that demonstrates many of the results in Structural Ramsey theory is
the Random (or Rado) graph.
It is an example of a Fraïssé limit, which is a construction found in
model theory.
Results from Structural Ramsey theory can then be used to state properties of
the Random graph.

This section will focus on the preliminaries behind the construction of Fraïssé
limits, the existence and uniqueness of the Random graph, and some of the
Ramsey type properties that it exhibits.

\subsection{Structures, Ages and Amalgamation Classes}
\label{sec:4.1}

We begin with a few preliminaries and some important results about them.

\begin{definition}[Structures over a Relational Language\supercite{Hubicka2015}]
  Let $I$ be a finite index set.
  We call $\Delta = (\delta_i \in \Z^+)_{i\in I}$ a
  type and the pair $L=(I,\Delta)$ a language.
  Given a set $A$, an $n$-ary relation is a subset of $A^n$.
  An $L$-structure $\str{A}$ is a pair $(A,(\rel{i}{\str{A}})_{i\in I})$ where each
  $\rel{i}{\str{A}}$ is a $\delta_i$-ary relation, i.e. a subset of $A^{\delta_i}$.

  If $A$ is finite, the structure $\str{A}$ is said to be finite, with a similar
  statement holding for countable sets.
  In general, the cardinality of a structure is defined to be the cardinality of
  the set on which it is defined.
\end{definition}

An example of a language is $L=(\{1\}, (2))$.
An example of a structure on that language is $\str{A} = (A, \rel{i}{A})$, where
$A$ is any non-empty finite subset $A \subset \Z^+$, equipped with the relation
$\rel{<}{A} = \{(x,y): x < y\} \subset A^2$. 
This forms a class of structures that we will call $\class{R}$, the class of
finite linear orderings.

\begin{definition}[Homomorphisms and Substructures\supercite{Hubicka2016,Hubicka2015}]
  Homomorphisms between structures over the same language are maps that preserve
  relations,\\
  so $f : \str{A} \to \str{B}, \rel{i}{\str{A}} \mapsto
  \rel{i}{\str{B}}, \forall i \in I$.

  An injective homomorphism where $f^{-1} : \rel{i}{\str{B}} \mapsto
  \rel{i}{\str{A}}, \forall i \in I$ is called an embedding, an isomorphism of
  structures is a surjective embedding.

  We say that $\str{B}$ is a substructure of $\str{A}$ if the inclusion $B
  \subseteq A$ is an embedding of structures.
  We denote the set of substructures of $\str{A}$ isomorphic to $\str{B}$ by
  $\sstr{A}{B}$.
\end{definition}

Returning to $\class{R}$, there are many bijections between subsets of $\Z^+$ of
the same cardinality, but this may not always preserve the relation.
To show that there is an isomorphism between any two $\str{A}, \str{B} \in
\class{R} : \card{\str{A}} = \card{\str{B}}$ one can proceed by induction (show
that there is an isomorphism between $\str{A} : \card{\str{A}} = n$ and
$\str{N}_n = (\{1,\dots,n\},\{(1,2),\dots,(1,n),(2,3),\dots,(n-1,n)\})$).
With this we can say that there are only countably many structures in
$\class{R}$ up to isomorphism.

$\class{R}$ is also an example of an amalgamation class, members of which
exhibit a few more properties.

\begin{definition}[Amalgamation Class\supercite{Cameron1990,Hubicka2016}]
  A class $\class{K}$ is an amalgamation class if it satisfies the following
  properties:
  
  \begin{itemize}
  \item \textbf{Hereditary Property}: For every $\str{A} \in \class{K}$ and every
    substructure $\str{B} \subseteq \str{A}, \str{B}\in\class{K}$ 
  \item \textbf{Joint Embedding Property}: For every $\str{A}, \str{B} \in
    \class{K}$, there exists $\str{C} \in \class{K}$ such that both $\str{A}$ and
    $\str{B}$ embed in $\str{C}$
  \item \textbf{Amalgamation Property}: For every $\str{A}, \str{B}_1, \str{B}_2
    \in \class{K}$ with $\str{A}$ embedding in both $\str{B}_1$ and $\str{B}_2$,
    there exists $\str{C} \in
    \class{K}$ such that $\str{B}_1$ and $\str{B}_2$
    both embed into $\str{C}$ and the diagram below commutes.
  \end{itemize}
  \begin{center}
    \adjustbox{scale=1.5}{%
      \begin{tikzcd}
        A
        \arrow[r, hookrightarrow, "\alpha_1"]
        \arrow[d, hookrightarrow, "\alpha_2"]&
        B_1\arrow[d, dashed, hookrightarrow, "\exists\beta_1"]\\
        B_2\arrow[r, dashed, hookrightarrow, "\exists\beta_2"]& C
      \end{tikzcd}
    }
  \end{center}
\end{definition}

\begin{example}
  $\class{R}$ is an amalgamation class

  \begin{proof}
    Prove properties, suppose $\str{B_1},\str{B_2}\in\class{R}$.
    \begin{itemize}
    \item\textbf{HP}: Given $B_1 \subseteq B_2$, we
      have $\card{\str{B_1}} \leq \card{\str{B_2}}$. Every $\str{A}\in\class{R} :
      \card{\str{A}} = m$ is isomorphic to $\str{N}_m$ as defined above and
      there is a clear embedding of $\str{N}_m$ into $\str{N}_n$ where $m \leq
      n$.
    \item\textbf{JEP}: Suppose again that $\card{\str{B_1}} = m,
      \card{\str{B_2}} = n$.
      $\str{B_1}$ embeds into $\str{N}_{m+n}$ by mapping to the values
      $1,\dots,m$ and $\str{B_2}$ also embeds by mapping to the values $m+1,\dots,n$.
    \item\textbf{AP}: Suppose that there is $\str{A}$ which embeds into both
      $\str{B_1}$ and $\str{B_2}$ by $\alpha_1, \alpha_2$ respectively.
      Partition $B_1, B_2$ into three classes each:
      \begin{itemize}
      \item The $x\in B_i$ such that $\exists y \in A$ with
        $(x,\alpha_i(y)) \in \rel{<}{B_i}$, call these sets $B^-_i$.
      \item The $x\in B_i$ such that $\exists y \in A$ with
        $x=\alpha_i(y)$, call these sets $B^A_i$.
      \item The $x\in B_i$ such that $\exists y \in A$ with
        $(\alpha_i(y),x) \in \rel{<}{B_i}$, call these sets $B^+_i$.
      \end{itemize}
      Define, $n_1 = \card{B^-_1}, n_2 = n_1 + \card{B^-_2}, n_3 = n_2 + \card{A},
      n_4 = n_3 + \card{B^+_2}, n_5 = n_4 + \card{B^+_2}$ and let $C = \str{N}_{n_5}$.
      We define the embeddings as follows: 
      \begin{align*}
        \beta_1:&\str{B_1}\to\str{C},\\
                &B^-_1\mapsto\{1,\dots,n_1\},\\
                &B^A_1\mapsto\{n_2+1,\dots,n_3\},\\
                &B^+_1\mapsto\{n_3+1,\dots,n_4\}\\
        \beta_2:&\str{B_2}\to\str{C},\\
                &B^-_2\mapsto\{n_1,\dots,n_2\},\\
                &B^A_2\mapsto\{n_2+1,\dots,n_3\},\\
                &B^+_2\mapsto\{n_4+1,\dots,n_5\}
      \end{align*}
      It is then clear that $\beta_1\circ\alpha_1 = \beta_2\circ\alpha_2$.
    \end{itemize}
  \end{proof}
\end{example}

Another useful construction is the age of a structure.

\begin{definition}[Age\supercite{Cameron1990,Macpherson2011}]
  The age of a countable structure, say $\str{A}$, is the class of finite
  $L$-structures, $\str{B}$, such that $\str{B}$ is isomorphic to
  a substructure of $\str{A}$.
\end{definition}

It is reasonably clear that $\age\str{A}$ has the hereditary property. 
% THIS IS WRONG
\textbf{WRONG -->}
$\age\str{A}$ also has the joint embedding property as $\str{B}_1, \str{B}_2
\in \age\str{A} \implies \str{B}_1, \str{B}_2$ embed in $\str{A} \in \age\str{A}$.

Two natural questions to ask now are whether the age of a particular structure
is an amalgamation class, and whether a given amalgamation
class is the age of some structure.
The correspondence between the two is given by Fraïssé's theorem which makes use
of the concept of homogeneity.

\begin{definition}[Homogeneous Structures\supercite{Macpherson2011}]
  A structure $\str{A}$ is homogeneous if for every pair of finite isomorphic
  substructures $\str{U}, \str{V} \subseteq \str{A}$, with $f : \str{U} \to
  \str{V}$ being the isomorphism, there exists an automorphism $\tilde{f}
  \in\aut\str{A}$ which extends $f$, i.e. $\tilde{f}$ restricted to $\str{U}$ is
  $f$.
\end{definition}

An example of homogeneity is given by the rationals with the order relation.
The argument below is taken from \textcite[1601]{Macpherson2011}.
\begin{example}
  $\str{Q} = (\mathbb{Q},<)$ is homogeneous.
  \begin{proof}
    Let $\str{U}, \str{V} \subseteq \str{Q}$ be an isomorphic pair of
    finite substructures with isomorphism $f$.
    Since they are both finite, they both have a minimum and maximum.
    Let $a \in \mathbb{Q}\setminus U$, since $\str{U}$ is finite either:
    \begin{enumerate}
    \item $\forall u \in U, a < u$, this is equivalent to $a < u\sumin$
    \item $\exists u_1, u_2 \in U, u_1 < a < u_2$
    \item $\forall u \in U, u < a$, this is equivalent to $u\sumax < a$
    \end{enumerate}
    Construct $\tilde{f}$ such that $\tilde{f} : (-\infty,u\sumin) \mapsto
    (-\infty,f(u\sumin)), (u_i,u_{i+1}) \mapsto (f(u_i),f(u_{i+1})),
    (u\sumax,\infty) \mapsto (f(u\sumax),\infty)$ where the bounded intervals
    are mapped via a linear interpolation and the unbounded intervals are mapped
    linearly.
    This is an automorphism on $\str{Q}$
  \end{proof}
\end{example}

We now state the main theorem of this subsection

\begin{theorem}[Fraïssé\supercite{Fraisse1953}]
  Let $L$ be a language
  \begin{itemize}
  \item Given a homogeneous $L$-structure $\str{M}$, $\age\str{M}$ is an
    amalgamation class.
  \item Given a countable non-empty amalgamation class of finite structures
    $\class{C}$, there exists a unique homogeneous $L$-structure $\str{M}$ such
    that $\age\str{M} = \class{C}$.
  \end{itemize}
\end{theorem}

We omit a proof of this theorem, but a sketch is given in \textcite[1602]{Macpherson2011}.
We call the structure given by the second point of the theorem the Fraïssé limit
of the class $\class{C}$.

A good example of a Fraïssé limit is that of $\class{R}$, which has limit $\str{Q}$.
It is clear that any finite linear order can be embedded into $\str{Q}$, so it
only remains to show that any finite substructure of $\str{Q}$ is isomorphic to
some structure in $\class{R}$.
This can be done by induction in a similar manner to showing that there are only
countably many structures in $\class{R}$ up to isomorphism.

\subsection{The Random Graph}
\label{sec:4.3}

We now focus on the existence, construction and uniqueness of the random graph.
The random graph is another example of a Fraïssé limit, but the amalgamation class in this
case is the class of finite graphs\cite{Cameron2013}.



\subsection{Ramsey Classes}
\label{sec:4.4}
The original statement of Ramsey's theorem is, broadly speaking, a statement
about the existence of a sub-object in any partition of a 'large-enough' object.
We now introduce the idea of a Ramsey class which applies this to classes of
structures.

\begin{definition}[Ramsey Class\supercite{Nesetril1995}]
  A structure $\str{C}$ is said to be $(t,\str{A})$ Ramsey for a structure
  $\str{B}$ if every $t$-colouring of the set $\sstr{C}{A}$ contains
  $\str{B'}: \str{C}\supseteq\str{B'} \simeq \str{B}$ such that $\sstr{B'}{A}$ is
  monochromatic.
  To make this more concise, we write $\str{C}\radoArrow{\str{B}}{\str{A}}{t}$.

  A class $\class{K}$ is said to have the $\str{A}$-Ramsey property if for every
  $\str{B}\in\class{K}$ and $t\in\Z^+$, there exists $\str{C}\in\class{K}$ such that
  $\str{C}\radoArrow{\str{B}}{\str{A}}{t}$.

  If $\class{K}$ has the $\str{A}$-Ramsey property for every $\str{A}\in
  \class{K}$, then $\class{K}$ is a Ramsey class.
\end{definition}

A concrete example can be given in terms of finite complete graphs.
The well-known fact that any 2-colouring of $K_6$ contains a monochromatic
$K_3$ translates to the statement $K_6\radoArrow{K_3}{K_2}{2}$.
Due to the generalised Ramsey theorem, we can state that the class of finite
complete graphs has $K_2$-Ramsey property.
In fact, a direct corollary of the finite Ramsey theorem from the original paper
\cite[267]{Ramsey1930} is that the class of finite complete graphs is a Ramsey
class.

% LocalWords:  Ramsey subgraphs subgraph Fraïssé surjective

% Local Variables:
% TeX-master: "m2r-report.tex"
% End:
