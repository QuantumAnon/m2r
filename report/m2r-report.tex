\documentclass{article}

\usepackage{parskip}
\usepackage{subfiles}

\usepackage{amssymb, amsmath, amsthm}
\usepackage{titling}

\usepackage{enumitem}

\usepackage{tikz}
\usepackage{adjustbox}
\usetikzlibrary{cd}

\usepackage[style=numeric]{biblatex}
\usepackage{hyperref}

\newtheoremstyle{BreakBold}
{5pt}% measure of space to leave above the theorem.
{}% measure of space to leave below the theorem.
{\itshape}% name of font to use in the body of the theorem
{}% measure of space to indent
{\bfseries}% name of head font
{\newline}% punctuation between head and body
{.5em}% space after theorem head
%; " " = normal inter-word space
% Manually specify theorem head
{\thmname{#1}\thmnumber{ #2}:\thmnote{ [#3]}}

\theoremstyle{BreakBold}
\newtheorem{theorem}{Theorem}[section]
\newtheorem{definition}[theorem]{Definition}
\newtheorem{lemma}[theorem]{Lemma}
\newtheorem{prop}[theorem]{Proposition}
\newtheorem*{cor}{Corollary}
\newtheoremstyle{example}{5pt}{}{}{}{\bfseries}{\newline}{.5em}{\thmname{#1}:\thmnote{ #3}}
\theoremstyle{example}
\newtheorem*{example}{Example}

\newcommand{\str}[1]{\mathbf{#1}}
\newcommand{\rel}[2]{R^{#1}_{\str{#2}}}
\newcommand{\class}[1]{\mathcal{#1}}
\newcommand{\Z}{\mathbb{Z}}
\newcommand{\sstr}[2]{\binom{\str{#1}}{\str{#2}}}
\newcommand{\ramsSplit}[3]{(#1)^{#2}_{#3}}
\newcommand{\radoArrow}[3]{\rightarrow\ramsSplit{#1}{#2}{#3}}
\newcommand{\card}[1]{|#1|}
\newcommand{\sumin}{_{\text{min}}}
\newcommand{\sumax}{_{\text{max}}}

\usepackage{geometry}
\geometry{
  margin=3cm
}

\DeclareMathOperator{\age}{Age}
\DeclareMathOperator{\aut}{Aut}

\bibliography{ref.bib}

\title{An Overview of Ramsey Theory}
\author{
  Chaudhuri, Gautam\\
  \and
  Kveder, Anthony\\
  \and
  Malbon, Joe\\
  \and
  Tarallo, Guglielmo\\
}

\begin{document}
\begin{titlingpage}
\maketitle
\begin{abstract}
  A problem that is often considered when studying a discrete structure is
  determining the existence of a substructure contained in one class of a partition.
  Ramsey theory is the study of a class of sufficient conditions for the
  existence of such a substructure.
  In this report we present several results from Ramsey theory with the goal of
  demonstrating the breadth of the subject area.
  We begin by providing an overview of the classical results from Ramsey in
  their graph theoretic formulation.
  We then proceed to analyse outstanding results from Hales and Jewett, Szemerédi,
  and van der Waereden.
  Next we show how an early generalization of results by van der Waerden and
  Rado leads to a geometric constraint on Ramsey configurations in Euclidean
  space.
  Finally, we generalise some results through the language of model theory and
  use them to construct and study the random graph.
\end{abstract}
\tableofcontents
  
\end{titlingpage}

% Section numbers are a rough guide, they can be switched around later
\section*{Introduction}
\label{sec:0}
\subfile{00-intro}

\section{Graph Theoretical Ramsey Theory}
\label{sec:1}
\subfile{01-graph-theory}

\section{The Hales-Jewett Theorem}
\label{sec:2}

\newpage
\section{Euclidean Ramsey Theory}
\label{sec:4}
\subfile{04-euclidean-ramsey-theory}

\newpage
\section{Structural Ramsey Theory}
\label{sec:4}
\subfile{02-structural-ramsey}

\printbibliography
\end{document}
% LocalWords:  Ramsey Jewett ramsey
