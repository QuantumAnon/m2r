\documentclass[./main.tex]{subfiles}

\subsection{Rado's Theorem}
\label{sec:4.1}

Here we denote, for $n\in\mathbb{N}$, $$[n]:=\{1,\cdots,n\}$$

The following results are commonly seen as precursors of Ramsey theory: Hilbert's cube Lemma (1892),
Schur's Theorem (1916) and van der Waerden's Theorem (1927). (Schacht2014)\\

A student of Schur's, Rado, generalised these results in 1933 (Schacht2014). 
In fact they can all be seen as statements abot monochromatic solutions to systems of linear homogeneous equations, in the sense that will be specified now.\\

Let $\mathcal{L}(x_{1},\cdots,x_{n})$ be a system of linear homogeneous equations in variables $ x_{1},\cdots,x_{n}$, where $\{x_{1},\cdots,x_{n}\}$ are assumed to be elements of a set $X$.\\

For some $k\in\mathbb{N}$ consider a $k$-colouring of $X$ (that is, any partition of $X$ into k equivalence classes). 
Then a solution $(x_{1}^*,\cdots,x_{n}^*)$ of  $\mathcal{L}$ is monochromatic if $\{x_{1}^*,\cdots,x_{n}^*\}$ is entirely contained in one of the $k$ equivalence classes.\\

Equivalently, let $\chi$ be a map:


$$\chi: X \to [k] $$


Then our $k$-colouring is the partition induced by the equivalence relation $R = \{(x,y) \in X| \chi(x) = \chi(y)\}$, and a solution $(x_{1}^*,\cdots,x_{n}^*)$ of  $\mathcal{L}$ is monochromatic if there exists $p\in [k] $ such that $\chi(\{x_{1}^*,\cdots,x_{n}^*\}) = \{p\}$.\\
More in general, $x =(x_{1},\cdots,x_{n}) \in X^n$ is monochromatic if there exists $p\in[k]$ with $ \chi(\{x_{1},\cdots,x_{n}\}) = \{p\}$\\

To formulate our theorem, we write the system $\mathcal{L}(x_{1},\cdots,x_{n})$ as:
$$\Lambda x = 0$$
where $\Lambda$ is an appropriate matrix and $x = (x_{1},\cdots,x_{n}) \in \mathbb{N}^n$.\\

We first introduce some definitions:\\

\begin{mydef}
A matrix $A \in \mathbb{Q}^{m\times n}$ is \underline{partition regular} (PR) if:
\begin{center}
$\forall r \in \mathbb{N}$, for all $r$-colouring of $\mathbb{N}$, there exists a monochromatic solution to $Ax = 0$ in $\mathbb{N}^n$ 
\end{center}
\end{mydef}

Let $A\in\mathbb{Q}^{m\times n}$.
It is easy to see that $A$ is PR if and only if $\lambda A$ is PR for all $\lambda\in\mathbb{Q}\setminus{\{0\}}$. Hence all our statements about PR matrices can be formulated, without loss of generality, for matrices with integer entries.\\

\begin{mydef}
A matrix $A = \begin{pmatrix}
{g_{1} & \cdots & g_{n}} 
\end{pmatrix}\in \mathbb{Q}^{m\times n}$ satisfies the \underline{columns condition} (CC) if $[n]$ can be partitioned into a family of sets $$\bigg\{\Gamma_{i}\bigg\}_{i \in [l]}$$ such that:

\begin{enumerate}
\item $\sum\limits_{j\in\Gamma_{1}}g_{j} = 0$
\item $\forall i \in [l]\setminus\{1\}$, $\sum\limits_{ j \in {\Gamma_{i}}}g_{j} \in span\bigg\{g_{j}|j\in\bigcup\limits_{k \in [l]: k < i }\Gamma_{k}\bigg\}$
\end{enumerate}

\end{mydef}

Rado's theorem states:\\

\begin{theorem}[Rado, 1933]
Let $A\in\mathbb{Q}^{m\times n}$. Then $A$ is partition regular if and only if it satisfies the columns condition.
\end{theorem}

We first prove the result for matrices of the form $ A = \begin{pmatrix}
{a_{1} & \cdots & a_{n}} 
\end{pmatrix} \in\mathbb{Q}^{1\times n}$.\\
Our strategy will be to find a property $\mathcal{E}$ which is equivalent to both the PR property and the CC property.(Liu2016)\\
The property is the following: $A = \begin{pmatrix}
{a_{1} & \cdots & a_{n}} 
\end{pmatrix}\in\mathbb{Q}^{1\times n}$ is $\mathcal{E}$ if there exists $\mathcal{I}\subset [n]$ such that: $$\sum\limits_{i\in\mathcal{I}} a_{i} = 0$$

It is easy to show that:\\

\begin{lemma}
$A$ is $\mathcal{E}$ if and only if $A$ satisfies CC.
\end{lemma}

For the equivalence of $\mathcal{E}$ and the PR property, we use the following lemma, which follows from van der Waerden's theorem:\\

\begin{lemma}
For all $k\in\mathbb{N}$ there exists $n\in\mathbb{N}$ such that, for any $k$-colouring of $[n]$, there exists a monochromatic $(x,y,z)$ satisfying: $$x+\lambda y = z$$
\end{lemma}

Using this we can prove that the $\mathcal{E}$ property is sufficient for A to be PR. Finally, it follows from elementary number theoretic results that the $\mathcal{E}$ property is necessary for A to be PR.\\

In this way we have proved Rado's theorem for matrices of the form $ A = \begin{pmatrix}
{a_{1} & \cdots & a_{n}} 
\end{pmatrix} \in\mathbb{Q}^{1\times n}$.
\end{mydef}

\section{Euclidean Ramsey theory}
\label{sec:4.2}

What makes Rado's theorem beautiful is its ability to translate a combinatorial property of the positive integers into an algebraic property of matrices. That is, it enables us to solve extremely general problems of the Hilbert-Schur-van der Waerden kind, through a unified algebraic approach.\\

In this section we will see that a generalization of Rado's theorem (Erdo\H s 1973) will enable us to translate a combinatorial property of finite subsets of the Euclidean space into a geometric property of the points of such set.



