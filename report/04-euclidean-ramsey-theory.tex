\documentclass[./m2r-report.tex]{subfiles}

\subsection{Rado's Theorem}
\label{sec:4.1}

In this section we show how a theorem from Rado (\cite{Rado1933}) generalising results
from arithmetical Ramsey theory is the key to proving an interesting connection
between combinatorial and algebraic properties of finite subsets of the 
Euclidean space. Now, Rado's theorem and its variants use purely Ramsey 
theoretical techniques in their proof. This illustrates once more how Ramsey theory is 
capable of translating combinatorial properties into structural (ie. geometric, 
algebraic, etc.) properties.

%Here we denote, for $n\in\mathbb{N}$, $$[n]:=\{1,\cdots,n\}$$
The following results are commonly seen as precursors of Ramsey theory: Hilbert's 
cube Lemma (1892), Schur's Theorem (1916) and van der Waerden's Theorem (1927) (\cite{Schacht2011}). 
They are all usually classified as arithmetical results, as they deal with 
combinatorial properties of $\mathbb{N}$.

Rado, who was a student of Schur's, generalised these results in 1933 (\cite{Schacht2011}). 
In fact they can all be seen as statements abot monochromatic solutions to 
systems of linear homogeneous equations, in the sense that will be specified now.

\begin{definition}[Colouring of a set]
Let $X$ be a non-empty set. For $k\in\mathbb{N}$, a $k$-colouring of $X$ is a 
partition of $X$ into $k$ equivalence classes. 
\end{definition}

Equivalently, a $k$-colouring of $X$ is the partition of $X$ induced by an 
arbitrary map: $$\chi: X \to [k]$$ through the equivalence relation: 
$$\mathcal{R}_{\chi}:= \{(x_{1},x_{2})\in X | \chi(x_{1})=\chi(x_{2}) \}$$
and $\chi$ is called the colouring function. Somewhat improperly, 
we will identify a $k$-colouring of $X$ with its colouring function.

\begin{definition}[Monochromatic vectors]
Let $X$ be non-empty and let $\chi$ be a $k$-colouring of $X$ for some 
$k\in\mathbb{N}$. Then an element $x = (x_{1},\cdots,x_{n})\in X^n$ is 
monochromatic if: 
$$\exists p \in [k] : \chi\bigg(\{x_{1},\cdots,x_{n}\}\bigg) = \{p\}$$
\end{definition}

In the following, we will be studying systems of linear equations with rational 
coefficients. These will be denoted $\mathcal{L }=\mathcal{L}(x_{1},\cdots,x_{n})$. 
We are interested in monochromatic solutions $x\in \mathbb{N}^n$ to $\mathcal{L}$. 
We can formulate our problem in the following way:

\begin{definition}[Partition Regular matrices]
Let $\Lambda\in\mathbb{Q}^{n\times m}$. $\Lambda$ is $k$-regular over 
$\mathbb{N}$ if, under any $k$-colouring of $\mathbb{N}$, there exists a 
monochromatic $x\in\mathbb{N}^n$ such that $\Lambda x  = 0$.
$\Lambda$ is Partition Regular (PR) over $\mathbb{N}$ if $\Lambda$ is 
$k$-regular over $\mathbb{N}$ for all $k$.
\end{definition}

It is easy to see that $\Lambda$ is PR if and only if $\mu\Lambda$ is PR for all 
$\mu\in\mathbb{Q}\setminus{\{0\}}$. Hence our statements about PR matrices can 
be formulated, without loss of generality, for matrices with integer entries.
% Notice also that partition regularity over $\mathbb{N}$ implies partition regularity over $\mathbb{Z}$, as any finite colouring of $\mathbb{Z}$ induces a finite colouring of $\mathbb{N}$.

We are interested in PR matrices as they allow us to frame and formulate problems 
of elementary arithmetical Ramsey theory in an enlightening and compact way. 
For example, Schur's theorem states that the matrix 
$\begin{pmatrix}
1  & 1 & -1
\end{pmatrix}$ is PR.
This turns out to be particularly useful since Rado's theorem establishes an 
equivalence between partition regularity and another, more algebraic property 
of matrices:

\begin{definition}[Columns Property]
A matrix $\Lambda = \begin{pmatrix}
g_{1} & \cdots & g_{n} 
\end{pmatrix}\in \mathbb{Q}^{m\times n}$ satisfies the Columns Property (CP) 
if $[n]$ can be partitioned into a family of sets 
$$\bigg\{\Gamma_{1},\cdots,\Gamma_{r}\bigg\}$$ such that:

\begin{enumerate}[label=(\roman*)]
\item $\sum\limits_{j\in\Gamma_{1}}g_{j} = 0$
\item $\forall i \in [r]\setminus\{1\}$, $\sum\limits_{ j \in {\Gamma_{i}}}g_{j} 
\in span\bigg\{g_{j} \ \bigg| \ j\in\bigcup\limits_{k \in [r]: k < i }\Gamma_{k}\bigg\}$
\end{enumerate}

\end{definition}

Rado's theorem states:

\begin{theorem}[Rado, 1933]
Let $\Lambda\in\mathbb{Q}^{m\times n}$. Then $\Lambda$ is partition regular if 
and only if it satisfies the columns property.
\end{theorem}

To prove Rado's theorem we use the following lemmas, the first of which results 
almost immediately by application of Gram-Schmidt orthogonalization:

\begin{lemma}
Let $\{v_{1},\cdots,v_{n},v\}\in \mathbb{Z}^m$. If $v \not\in span\{v_{1},\cdots,v_{n}\}$, 
then there exists $u\in\mathbb{Z}^m$ such that: $u.v\not = 0$ and $\forall i\in [n], u.v_{i}=0$.
\end{lemma}

To state our next Lemma, we need a definition:

\begin{definition}[(m,p,c)-sets]
Let $m,p,c \in \mathbb{N}$. Then we define:
$$N_{m,p,c}:=\bigg\{\mu = (\mu_{1},\cdots,\mu_{m+1}) \  \bigg|  \ \exists j 
\in [m]: \forall i<j, \ \mu_{i}= 0 \ \emph{and} \ \mu_{j}=c \ \emph{and}  
\ \forall i>j, \ |\mu_{i}|\leq p\bigg\} $$
and for a given generator $y\in \mathbb{N}^{m+1}$ the (m,p,c)-set is defined as:
$$S_{m,p,c}(y):=\bigg\{y.\mu \ | \mu\in N_{m,p,c}\bigg\}$$
\end{definition}



\begin{lemma}
Let $m,p,c \in \mathbb{N}$. For any finite colouring of $\mathbb{N}$, 
there exists $y\in \mathbb{N}^{m+1}$ such that $S_{m,p,c}(y)$ is monochromatic.
\end{lemma}


This is proved through application of van der Waerden.


We now sketch a proof of Rado's theorem. The proof is taken from Liu (see \cite{Liu2016}).


\textit{Sketch of proof of Rado}.
We work with $\Lambda\in\mathbb{Z}^{m\times n}$ without loss of generality.

For the necessary implication, we choose the following colouring: for an arbitrary 
prime $p$, let $\chi_{p}:\mathbb{N}\to[p-1]$ such that, if $y\in\mathbb{N}$ is 
written in basis $p$, its colour is the least non-zero coefficient $d(y)$. 
Call $\pi(y)$ its position in the expansion. Since $\Lambda$ is PR, there exists 
a monochromatic solution $x = (x_{1},\cdots,x_{n})\in \mathbb{N}^n$ such that 
$\Lambda x = 0$. Now order the natural numbers $x_{1},\cdots,x_{n}$ in an 
increasing fashion by position $\pi(x_{j})$ of the first non-zero coefficient 
$d(x_{j})$ in their $p$-expansion. Partition their indices into sets 
$\Gamma_{i}:=\{j\in [n] \ | \ \pi(x_{j})=i\}$. Then by means of properties of 
primes integers and our first Lemma one shows that such a partition satisfies 
the Columns Property conditions.

For the sufficient implication, we show that there exist $m,p,c\in\mathbb{N}$ 
such that for all $x\in\mathbb{N}$, $S_{m,p,c}(x)$ contains a solution to our 
system of equations. This is done constructively, using the Columns Property 
that we assume. Then, by our second Lemma, one such solution must be monochromatic. 
$\qed$

\subsection{Euclidean Ramsey theory}
\label{sec:4.2}

Rado's theorem is beautiful because it translates a combinatorial 
property of the positive integers into an algebraic property of matrices. 
That is, it enables us to solve general problems of the Hilbert-Schur-van 
der Waerden kind, through a unified algebraic approach. It also provides us 
with an algebraic constraint on Ramsey sets in $\mathbb{Z}$.

In this section show how a generalization of Rado's theorem (\cite{Erdös1973}) 
enables us to translate a combinatorial property of finite subsets of the Euclidean 
space into a geometric property. In particular we identify a geometrical necessary 
condition for a finite set to be Ramsey in $\mathbb{R}^n$.

We first define a notion of Ramseyness for finite sets $F\in\mathbb{R}^n$. 

\begin{definition}[Configurations]
A figure is a finite set $F\in \mathbb{R}^m$.
% for some $m\in \mathbb{N}$ such that the elements of K are subject to purely 
%metrical costraints.
Consider the set of all figures 
$$\mathfrak{F}:= \{F \ | \ \exists n \in \mathbb{N} : F \  
\textit{is finite in} \ \mathbb{R}^n\} $$ 
This is clearly non-empty as $\{0\}\subset\mathbb{R}$ is in $\mathfrak{F}$.
Define an equivalence relation as follows: $$F \sim F' 
\iff \ \textit{there exists an isometry} \ \Psi \ \textit{such that} 
\ \Psi(F)=F'$$
Then we define the set of Euclidean configurations 
$$\mathfrak{K}:=\frac{\mathfrak{F}}{\sim}$$
An element $K=[F]$ of $\mathfrak{K}$ is called a configuration.
If $F$ and $F'$ are figures with $[F]=[F']$, we say that they have the same 
configuration.
\end{definition}

\begin{definition}[Ramsey property]
A configuration $K$ is Ramsey if: 
\begin{center}
$\forall r\in\mathbb{N}$, $\exists n\in\mathbb{N}$ such that, 
under any $r$-colouring of $\mathbb{R}^n$,  there exists a monochromatic 
$F\in K$ in $\mathbb{R}^n$
\end{center}
\end{definition}

A natural question to consider is whether there exists any class of 
configurations that are not Ramsey. Consider the following geometrical 
property of a figure:

\begin{definition}[Spherical figures]
A figure $F\in\mathbb{R}^m$ is spherical if there exists a sphere 
$S\in\mathbb{R}^m$ such that $F\subset S$.
\end{definition}

It is easy to see that if a figure $F$ is spherical, then any figure with 
the same configuration will be spherical. In fact isometries preserve spheres. 
Hence it makes sense to talk about spherical configurations.

Our aim in this section is to prove an interesting connection between Ramseyness
and the property of being spherical:

\begin{theorem}[Erd\H os,1973]
If $K$ is Ramsey, $K$ is spherical.
\end{theorem}

We will use the following characterisation of a spherical figure:

\begin{lemma}
A figure $F = \{v_{0},\cdots,v_{k}\}$ is not spherical if and only if there 
exist scalars $\{0\}\neq\{\mu_{1},\cdots, \mu_{k}\}\subset\mathbb{R}$ such that:

\begin{enumerate}[label=(\roman*)]
\item $\sum_{i=1}^{k} \mu_{i} (v_{i} - v_{0})= 0$;
\item $\sum_{i=1}^{k} \mu_{i} (|v_{i}|^2 - |v_{0}|^2) = b \neq 0 $
\end{enumerate}
\end{lemma}

\textit{Proof}. Assume that $F$ is spherical and (i) holds. Then, $\forall 
i \in \{0,\cdots,k\}$, $$|v_{i}|^2 - |v_{0}|^2 = |v_{i} - z|^2 - |v_{0} - z|^2 + 
2 \langle v_{i} - v_{0}, z \rangle = r^2 - r^2 + 2 \langle v_{i} - v_{0}, z \rangle 
= 2 \langle v_{i} - v_{0}, z \rangle $$ where $z$ and $r$ are respectively the 
centre and the radius of our sphere. But then:
$$\sum_{i=1}^{k} \mu_{i} (|v_{i}|^2 - |v_{0}|^2) = \sum_{i=1}^{k} 2\mu_{i} 
\langle v_{i} - v_{0}, z \rangle = 2  
\bigg\langle \sum_{i=1}^{k} \mu_{i} (v_{i} - v_{0}), z \bigg\rangle \stackrel{(i)}{=} 
2 \langle 0, z \rangle = 0$$
Hence (ii) is not true. Thus if (i) and (ii) hold, $F$ is not spherical.

Now assume $F$ is not spherical. Then (i) must hold; otherwise the elements of $F$
are in general position and hence $F$ is spherical. Now assume without loss of 
generality that $F$ has a spherical proper subset. In particular let $\mu_{k}\neq 0$
and consider the spherical subset $\{v_{0},\cdots,v_{k-1}\}$ with radius $r$ and 
centre $z$. Then:

$$\sum_{i=1}^{k} \mu_{i} (|v_{i}|^2 - |v_{0}|^2) \stackrel{(i)}{=} \mu_{k} (|v_{k} - z|^2 
- |v_{0} - z|^2)\neq 0 $$ 

As otherwise $F$ is spherical. So (ii) holds $\qed$

Notice that if $F$ is not spherical, then any other figure with the same 
configuration is not spherical. In particular, if $F$ has the same configuration 
as $E$, then the elements of $E$ satisfy our last Lemma for the same choice 
of scalar coefficients $\{\mu_{1},\cdots, \mu_{k}\}\subset\mathbb{R}$, as 
the equations (1) and (2) are invariant under isometry in $\mathbb{R}^n$.


Our main tool in the proof is the following Lemma, which is a generalization of Rado's theorem.
In fact it builds up on later results from Rado (see \cite{Rado1945}).  
The proof that follows is taken from Graham (see \cite{Graham1981}). Erd\H os gives a more general proof 
involving arbitrary fields.

\begin{lemma}[Erd\H os,1973]
Let $\mu_{1},\cdots,\mu_{k}$ and $b\neq 0$ be elements of $\mathbb{R}$. Then there 
exists a colouring $\chi$ of $\mathbb{R}$ such that the equation: 
$$\sum_{i = 1}^{k} \mu_{i}(x_{i} - x_{0}) = b \ \ \ \ \ \ \ \ (A) $$ has no monochromatic solution.
\end{lemma}

\textit{Proof}. I. We first show that there exists a $2n$-colouring $\chi$ of 
$\mathbb{R}$ such that the equation $$\sum_{i = 1}^{k} (x_{i} - x_{i}') 
= 1 \ \ \ \ \ \ \ \ (B)$$ 
has no solution with $\chi(x_{i}) = \chi(x_{i}')$ for all $i\in [k]$.

Choose the following colouring: $$\chi: \mathbb{R} \to [2n] \ | \ x \mapsto j \ \ \
\textit{if} \ \ \ \exists m \in \mathbb{Z} \ : \ x \in \bigg[2m + \frac{j}{n}, 
2m + \frac{j+1}{n}\bigg]$$
Notice that if $\chi(x) = \chi(y)$, then $\exists m \in \mathbb{Z} : x - y =
2m + \epsilon$ for some $\epsilon$ with $|\epsilon| < \frac{1}{n}$.
Now assume that there exists a solution of (B) with $\chi(x_{i}) = \chi(x_{i}')$ 
for all $i\in [k]$. Then there exist integers $m_{i}$ and real $\epsilon_{i}$ such that:
$$1 = \sum_{i = 1}^{k} (x_{i} - x_{i}') = \sum_{i = 1}^{k} (2m_{i} + \epsilon_{i})
= 2M + \sum_{i = 1}^{k} \epsilon_{i} $$
Now, $2M = \sum_{i = 1}^{k} 2m_{i}$ is an integer, whereas $|\sum_{i = 1}^{k} 
\epsilon_{i}| < 1$. But then $1 \notin \mathbb{Z}$, which is absurd.
\\

II. We now prove that there exists a colouring $\overline{\chi}$ of $\mathbb{R}$ such 
that (A) has no monochromatic solution. 
%with $\overline{\chi}(x_{i}) = \overline{\chi}(x_{i}')$ for all $i\in [k]$.
Notice that if $\overline{\mu}_{i} := \frac{\mu_{i}}{b}$, then (A) holds if and only if:
$$\sum_{i = 1}^{k} \overline{\mu_{i}}(x_{i} - x_{0}) = 1 \ \ \ \ \ \ \ \ (C)$$

Now consider the $2n$-colouring previously defined, and define a new colouring
$\overline{\chi}$ of $\mathbb{R}$ as follows:
\begin{center}

$\overline{\chi}(x) = \overline{\chi}(y)$ \ \ if \ \ $\forall i \in [k]$, \
$\chi(\overline{\mu}_{i} x) = \chi(\overline{\mu}_{i}y)$
\end{center}

Now assume there exists a monochromatic solution to (A) under this colouring. 
Then there exists a solution to (C) such that $\overline{\chi}(x_{i}) = 
\overline{\chi}(x_{0})$ for all $i \in [k]$. That is, 
$$\forall i \in [k], \ \ \chi(\overline{\mu}_{i} x_{i}) = \chi(\overline{\mu}_{i} x_{0})$$

and thus: $$ \forall i \in [k], \ \ \exists m_{i}, \epsilon_{i} \ : \ 
\overline{\mu}_{i} x_{i} - \overline{\mu}_{i} x_{0} = 2m_{i} + \epsilon_{i}$$ 

Where the $m_i$,$\epsilon_{i}$ satisfy the same properties as before for $i\in [k]$.
But then, by the same reasoning as before, (C) implies that $1\not\in\mathbb{Z}$,
which is absurd.

Thus (A) has no monochromatic solution under $\overline{\chi}$. $\qed$

\textit{Proof of Erd\H os}. Let $F$ be a non-spherical figure. Then there exist 
coefficients $\{\mu_{1},\cdots, \mu_{k}\}\subset\mathbb{R}$ such that equations 
(i) and (ii) from our second Lemma are satisfied. By our first Lemma there exists 
an $r$-colouring $\chi$ of $\mathbb{R}$ such that (ii) has no monochromatic 
solution. For $n \in \mathbb{N}$ we define the following colouring of 
$\mathbb{R}^n$: $$\chi_{n}^*: \mathbb{R}^n\to [r] : v \mapsto \chi(|v|^2) $$
Now, if there exists a monochromatic $E\in [F]$ in $\mathbb{R}^n$, then there 
exists a monochromatic solution to (ii), since the elements of $E$ satisfy (ii) with the same choice of coefficients by a 
previous observation. But this contradicts our first Lemma.

Thus there exists an $r\in \mathbb{N}$ such that for all $n \in \mathbb{N}$ 
there exists an $r$-colouring of $\mathbb{R}^n$ under which no element of $[F]$
in $\mathbb{R}^n$ is monochromatic. That is, $[F]$ is not Ramsey.\qed

% Local Variables:
% TeX-master: "m2r-report.tex"
% End: