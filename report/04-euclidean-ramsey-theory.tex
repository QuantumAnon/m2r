\documentclass[./main.tex]{subfiles}

\subsection{Rado's Theorem}
\label{sec:4.1}

The following results are commonly seen as precursors of Ramsey theory: Hilbert's cube Lemma (1892),
Schur's Theorem (1916) and van der Waerden's Theorem (1927). (Schacht2014)

A student of Schur's, Rado, generalised these results in 1933 (Schacht2014). 
In fact they can all be seen as statements abot monochromatic solutions to systems of linear homogeneous equations, in the sense that will be specified now.

Let $\Lambda(x_{1},\cdots,x_{n})$ be a system of linear homogeneous equations in variables $ x_{1},\cdots,x_{n}$, where $\{x_{1},\cdots,x_{n}\}$ are assumed to be elements of a set $X$.

For some $k\in \mathbb{N}$ consider a $k$-colouring of $X$ (that is, any partition of $X$ into k equivalence classes). 
Then a solution $(x_{1}^*,\cdots,x_{n}^*)$ of  $\Lambda$ is monochromatic if $\{x_{1}^*,\cdots,x_{n}^*\}$ is entirely contained in one of the $k$ equivalence classes.

\section{Euclidean Ramsey theory}
\label{sec:4.2}



