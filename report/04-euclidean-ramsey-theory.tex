\documentclass[./main.tex]{subfiles}

\subsection{Rado's Theorem}
\label{sec:4.1}

The following results are commonly seen as precursors of Ramsey theory: Hilbert's cube Lemma (1892),
Schur's Theorem (1916) and van der Waerden's Theorem (1927). (Schacht2014)

A student of Schur's, Rado, generalised these results in 1933 (Schacht2014). 
In fact they can all be seen as statements abot monochromatic solutions to systems of linear homogeneous equations, in the sense that will be specified now.

Let $\Lambda(x_{1},\cdots,x_{n})$ be a system of linear homogeneous equations in variables $ x_{1},\cdots,x_{n}$, where $\{x_{1},\cdots,x_{n}\}$ are assumed to be elements of a set $X$.

For some $k\in \mathbb{N}$ consider a $k$-colouring of $X$ (that is, any partition of $X$ into k equivalence classes). 
Then a solution $(x_{1}^*,\cdots,x_{n}^*)$ of  $\Lambda$ is monochromatic if $\{x_{1}^*,\cdots,x_{n}^*\}$ is entirely contained in one of the $k$ equivalence classes.

Equivalently, let $\chi$ be a map:


$$\chi: X \to \{1, \cdots, k\}$$


Then our $k$-colouring is the partition induced by the equivalence relation $R = \{(x,y) \in X| \chi(x) = \chi(y)\}$, and a solution $(x_{1}^*,\cdots,x_{n}^*)$ of  $\mathcal{L}$ is monochromatic if there exists $p\in \{1, \cdots, k\}$ such that $\chi(\{x_{1}^*,\cdots,x_{n}^*\}) = \{p\}$.

To formulate our theorem, we write the system $\mathcal{L}(x_{1},\cdots,x_{n})$ as:
$$\Lambda x = 0$$
where $\Lambda$ is an appropriate matrix and $x = (x_{1},\cdots,x_{n}) \in \mathbb{R}^n$.

We first introduce some definitions:

\begin{mydef}
A matrix $A \in \mathbb{R}^{m\cdot n}$ is \underline{regular} if:
\begin{center}
$\forall k \in \mathbb{N}$, for all $r$-colouring of $\mathbb{R}^n$, there exists a monochromatic solution to $Ax = 0$ 
\end{center}
\end{mydef}

\begin{mydef}
A matrix $A \in \mathbb{R}^{m\cdot n}$ satisfies the \underline{column condition} if its columns $\{g_{1},\cdots,g_{n}\}$ can be partitioned into a family of sets $$\bigcup\limits_{i \in \mathcal{I}}\Gamma_{i}$$ where $\mathcal{I}$ is a finite index set, which satisfies:

\begin{enumerate}
\item $\sum\limits_{g_{j} \in \Gamma_{1}}g_{j} = 0$
\item $\forall i \in \mathcal{I}\setminus\{1\}$, $\sum\limits_{g_{j} \in {\Gamma_{i}}}g_{j} \in span\bigg\{\bigcup\limits_{k \in \mathcal{I} : k < i }\Gamma_{k}\bigg\}$
\end{enumerate}

\end{mydef}

\section{Euclidean Ramsey theory}
\label{sec:4.2}



