\documentclass[./m2r-pres.tex]{subfile}

%A problem that is often considered when studying a discrete structure is determining the
%existence of a substructure contained in one class of a partition. Ramsey theory is the study of
%a class of sufficient conditions for the existence of such a substructure. In this presentation we present
%several results from Ramsey theory with the goal of demonstrating the breadth of the subject
%area.





%We begin by providing an overview of the classical results from Ramsey in their graph
%theoretical formulation. We then proceed to analyse outstanding results from van der Waerden,
%Hales and Jewett, and Szemerédi. Next we show how Rado’s generalisation of van der Waerden’s
%results inspires a theorem about geometrical constraints on Ramsey configurations in Euclidean
%space. Finally, we generalise some results through the language of model theory and use them to
%construct and study the random graph.