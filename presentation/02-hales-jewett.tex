\documentclass[./m2r-pres.tex]{subfile}

\begin{frame}
\frametitle{The Hales-Jewett Theorem}
\begin{itemize}[+]
\item<1-> $A = \{1, 2, ..., n\}$, $\Omega_{A, k} = \{(a_1, a_2, ..., a_k) : a_i \in A\}$
\item<2-> Roots: $\tau = (a_1, a_2, ..., a_{i-1}, x, a_{i+1}, ..., a_k) \in \Omega_{A \bigcup \{x\}, k} $
\item<3-> $\tau(a) = (a_1, a_2, ..., a_{i-1}, a, a_{i+1}, ..., a_k) \in\Omega_{A \bigcup \{x\}, k} $
\item<4-> $\tau = (1, 3, x, x), \tau(2) = (1, 3, 2, 2)$
\item<5-> Combinatorial line: $\mathcal{L}_{\tau} = \{\tau(a) : a \in A\} \subset \Omega_{A, k}$
\item<6-> e.g. $\{(1, 3, 1, 1), (1, 3, 2, 2), (1, 3, 3, 3)\}$
\end{itemize}

\end{frame}

\begin{frame}
\frametitle{The Hales-Jewett Theorem}

\begin{theorem} Given any size n of an alphabet $A = \{1, 2, ..., n\}$ and
  number of colours r, there exists an $N$ such that for any r-colouring of the
  set of N-words $\Omega_{A, N}$, there exists a monochromatic combinatorial
  line.
\end{theorem}
\begin{itemize}
\item<2-> Implies the Van der Waerden theorem: given any k and number of colours
  r, there exists an $N$ such that for any r-colouring of the set $\{1, 2, ...,
  N\}$, there is an monochromatic arithmetic sequence $\{a, a + d, ..., a +
  (k-1)d\}$ of length k.
\item<3-> $f: (a_1, a_2, ..., a_N) \mapsto a_1 + a_2 + ... + a_N$,\\
  $(a_1, a_2, ... x, ..., x, ..., a_N) \mapsto a_1 + a_2 + ... + a_N + 2x$
\end{itemize}
\end{frame}

\begin{frame}
\frametitle{The Density Hales-Jewett Theorem}

\begin{theorem} Given any size n of an alphabet $A = \{1, 2, ..., n\}$ and real
  number $\delta$, there exists an $N$ such that within any subset $B \subset
  \Omega_{A, N}$ of density $\delta$ there exists a combinatorial line.
\end{theorem}
\begin{itemize}
\item<2-> Implies Szemeredi's theorem, a density analogue of Van der Waerden's
  theorem.
\end{itemize}
\end{frame}
