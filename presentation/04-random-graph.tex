\documentclass[./m2r-pres.tex]{subfile}

\begin{frame}
  \frametitle{What is the Random Graph?}

  \begin{itemize}
  \item<2-> Let $\str{G}=(G,E_G)$ be a (countable) graph.
    \begin{remark}[Extensibility Property]
      $\str{G}$ has the extensibility property if for every finite $U, V
      \subsetneq G$ disjoint, non-empty, $\exists z \in G$ s.t. $z \in
      \adj U, z \notin \adj V$.
    \end{remark}
  \item<3-> The EP determines a structure up to isomorphism.
  \item<4-> If $\str{G}$ is s.t. $\forall x,y \in V,\Pr((x,y)\in
    E)=\frac{1}{2}$, then $G$ almost certainly has the extensibility property.
  \item<5-> Call this structure the Random Graph, $\rand$.
  \end{itemize}
\end{frame}

\begin{frame}
  \frametitle{Properties of the Random Graph}

  \begin{itemize}
  \item<2-> Homogeneity: If $\str{U}, \str{V}\subsetneq\rand$ finite, induced
    subgraphs s.t. $U\overset{f}{\cong} V$.
    Then, $\exists \tilde{f}\in\aut\rand$ that extends $f$.
  \item<3-> $\rand$ is a Fraïssé limit: $\rand$ is the unique countable
    structure s.t.\ every finite simple graph embeds into $\rand$.
  \item<4-> Does $\gra$ have any Ramsey style properties?
  \end{itemize}
\end{frame}

\begin{frame}
  \frametitle{Ramsey Classes}
  \begin{definition}
    $\class{K}$ has the $\str{A}\in\class{K}$\textit{-Ramsey property} if
    $\forall\str{B}\in\class{K}, \forall t\in\Z^+$, there is
    $\str{C}\in\class{K}$ s.t.\ every $t$-colouring of the $\str{A}$
    substructures of $\str{C}$ has a monochromatic $\str{B}$.

    $\class{K}$ is a \textit{Ramsey class} if it has the $\str{A}$-Ramsey property
    $\forall \str{A}\in\class{K}$.
  \end{definition}
  \begin{itemize}
    \item<2-> $\gra$ \textit{is not} a Ramsey class
    \item<3-> $\ogra$ \textit{is} a Ramsey class
    \item<4-> What can we say about $\orand$?
  \end{itemize}
\end{frame}

\begin{frame}
  \frametitle{$\aut\orand$}

  \begin{itemize}
  \item<2-> 
  \end{itemize}
  
\end{frame}
% LocalWords:  Ramsey
