\documentclass[./m2r-pres.tex]{subfile}

\begin{frame}
\frametitle{RADO'S THEOREM}

\begin{definition}[Partition Regularity]

  Let $\Lambda\in\mathbb{Q}^{m\times n}$. 
  
  We say that $\Lambda$ is \emph{$k$-regular} over 
$\mathbb{N}$ if, under any $k$-colouring of $\mathbb{N}$, there exists a 
monochromatic $x\in\mathbb{N}^n$ such that $\Lambda x  = 0$.


Then the matrix $\Lambda$ is \emph{partition regular} (PR) over 
$\mathbb{N}$ if $\Lambda$ is $k$-regular over $\mathbb{N}$ for all $k$.

\end{definition}

\end{frame}

\begin{frame}
  \frametitle{RADO'S THEOREM}
  
\begin{definition}[Columns Property]
  
  A matrix $\Lambda = \begin{pmatrix}
    g_{1} & \cdots & g_{n} 
    \end{pmatrix}\in \mathbb{Q}^{m\times n}$ satisfies the \emph{columns property} (CP) 
    if $[n]$ can be partitioned into a family of sets 
    $$\Pi := \bigg\{\mathcal{C}_{1},\cdots,\mathcal{C}_{r}\bigg\}$$ such that:
    
    \begin{itemize}
    \item $\sum\limits_{j\in\mathcal{C}_{1}}g_{j} = 0$
    \item $\forall i \in [r]\setminus\{1\}$, $\sum\limits_{ j \in {\mathcal{C}_{i}}}g_{j} 
    \in span \bigg\{g_{j} \ \bigg| \ j\in\bigcup\limits_{k \in [r]: k < i }\mathcal{C}_{k}\bigg\}$
    \end{itemize}


\end{definition}
\end{frame}


\begin{frame}{EUCLIDEAN RAMSEY THEORY}

  \begin{definition}[Euclidean Ramsey Sets]
    A finite set $F \in \mathbb{R}^m$ is \emph{Ramsey} if: 
    \begin{center}
    $\forall r\in\mathbb{N}$, $\exists n\in\mathbb{N}$ such that, 
    under any $r$-colouring of $\mathbb{R}^n$,  there exists a monochromatic 
    $E$ in $\mathbb{R}^n$ isometric to $F$.
    \end{center}
  \end{definition}
  

  
  \begin{definition}[Spherical sets]

  A set $F\in\mathbb{R}^m$ is spherical if there exists a sphere 
  $S\in\mathbb{R}^m$ such that $F\subset S$. That is, there exist $z\in\mathbb{R}^m$
  and $r\in\mathbb{R}^{+}$ such that: $$\forall x\in F, \ \ |x - z| = r$$
  where $|\cdot|$ is the Euclidean norm in $\mathbb{R}^m$.
  
  \end{definition}


\end{frame}